\documentclass[10pt, letterpaper]{article}

% Packages:
\usepackage[
    ignoreheadfoot,
    top=2 cm,
    bottom=2 cm,
    left=2 cm,
    right=2 cm,
    footskip=1.0 cm
]{geometry}
\usepackage{titlesec}
\usepackage{tabularx}
\usepackage{array}
\usepackage[dvipsnames]{xcolor}
\definecolor{primaryColor}{RGB}{0, 0, 0}
\usepackage{enumitem}
\usepackage{fontawesome5}
\usepackage{amsmath}
\usepackage[
    pdftitle={CV de Matheus Bruno de Moraes},
    pdfauthor={Matheus Bruno de Moraes},
    colorlinks=true,
    urlcolor=primaryColor
]{hyperref}
\usepackage[pscoord]{eso-pic}
\usepackage{calc}
\usepackage{bookmark}
\usepackage{lastpage}
\usepackage{changepage}
\usepackage{paracol}
\usepackage{ifthen}
\usepackage{needspace}
\usepackage{iftex}

% Ensure that generate pdf is machine readable/ATS parsable:
\ifPDFTeX
    \input{glyphtounicode}
    \pdfgentounicode=1
    \usepackage[T1]{fontenc}
    \usepackage[utf8]{inputenc}
    \usepackage{lmodern}
\fi

\usepackage{charter}

% Some settings:
\raggedright
\AtBeginEnvironment{adjustwidth}{\partopsep0pt}
\pagestyle{empty}
\setcounter{secnumdepth}{0}
\setlength{\parindent}{0pt}
\setlength{\topskip}{0pt}
\setlength{\columnsep}{0.15cm}
\pagenumbering{gobble}

\titleformat{\section}{\needspace{4\baselineskip}\bfseries\large}{}{0pt}{}[\vspace{1pt}\titlerule]

\titlespacing{\section}{
    -1pt
}{
    0.3 cm
}{
    0.2 cm
}

\renewcommand\labelitemi{$\vcenter{\hbox{\small$\bullet$}}$}
\newenvironment{highlights}{
    \begin{itemize}[
        topsep=0.10 cm,
        parsep=0.10 cm,
        partopsep=0pt,
        itemsep=0pt,
        leftmargin=0 cm + 10pt
    ]
}{
    \end{itemize}
}

\newenvironment{onecolentry}{
    \begin{adjustwidth}{
        0 cm + 0.00001 cm
    }{
        0 cm + 0.00001 cm
    }
}{
    \end{adjustwidth}
}

\newenvironment{header}{
    \setlength{\topsep}{0pt}\par\kern\topsep\centering\linespread{1.5}
}{
    \par\kern\topsep
}

\let\hrefWithoutArrow\href
\begin{document}

    \begin{header}
        {\fontsize{25pt}{25pt}\selectfont Matheus Bruno de Moraes}

        \vspace{5pt}

        \normalsize
        \mbox{Brazil} \textbar{} \mbox{\href{mailto:mbrunomoraes97@gmail.com}{mbrunomoraes97@gmail.com}} \textbar{} \mbox{\href{tel:+55(41)995757856}{+55 (41) 9 9575-7856}} \textbar{} \mbox{\href{https://linkedin.com/in/LinkedIn}{linkedin.com/in/LinkedIn}} \textbar{} \mbox{\href{https://github.com/Github}{github.com/Github}} \textbar{} \mbox{\href{https://}{}}
    \end{header}

    \vspace{5pt - 0.3cm}
    
\section{Resumo Profissional}
\begin{onecolentry}Engenheiro de Suporte e DevOps com experiência em infraestrutura Linux, CI/CD, APIs e automações. Participou ativamente da implementação de pipelines com GitHub Actions, conteinerização com Docker, orquestração com Kubernetes e resolução de problemas técnicos complexos. Forte habilidade em comunicação com times globais, documentação técnica e atendimento técnico ao cliente.\end{onecolentry}
\section{Experiência Profissional}

    \begin{onecolentry}
        \setcolumnwidth{\fill, 4.5cm}
        \begin{paracol}{2}
            \textbf{DevOps Engineer}, FlirT (Early Stage Startup) -- Remoto
            \switchcolumn
            \raggedleft 2025-Atual
        \end{paracol}
    \end{onecolentry}
    \vspace{0.10cm}
    \begin{onecolentry}
        \begin{highlights}
                    \item Gerenciou infraestrutura de aplicações em servidores Linux com automações baseadas em GitHub Actions.
                \item Implantou pipelines CI/CD e ambientes segregados de staging e produção utilizando Docker e Git.
        \end{highlights}
    \end{onecolentry}
    
\vspace{0.2cm}

    \begin{onecolentry}
        \setcolumnwidth{\fill, 4.5cm}
        \begin{paracol}{2}
            \textbf{Technical Onboarding Manager}, IREV -- Remoto – Limassol, Chipre
            \switchcolumn
            \raggedleft 2024-Atual
        \end{paracol}
    \end{onecolentry}
    \vspace{0.10cm}
    \begin{onecolentry}
        \begin{highlights}
                    \item Coordenou a integração técnica de clientes e investigou falhas em APIs e S2S com times de engenharia.
                \item Padronizou processos internos e contribuiu para conteúdo técnico de onboarding.
        \end{highlights}
    \end{onecolentry}
    
\vspace{0.2cm}

    \begin{onecolentry}
        \setcolumnwidth{\fill, 4.5cm}
        \begin{paracol}{2}
            \textbf{L2 Technical Support Engineer & Automation Specialist}, Multilogin -- Remoto – Tallinn, Estônia
            \switchcolumn
            \raggedleft 2023-2024
        \end{paracol}
    \end{onecolentry}
    \vspace{0.10cm}
    \begin{onecolentry}
        \begin{highlights}
                    \item Desenvolveu scripts de automação com Selenium, Playwright e Puppeteer para clientes corporativos.
                \item Documentou soluções técnicas e produziu vídeos para base de conhecimento.
        \end{highlights}
    \end{onecolentry}
    
\vspace{0.2cm}

    \begin{onecolentry}
        \setcolumnwidth{\fill, 4.5cm}
        \begin{paracol}{2}
            \textbf{Software Engineer}, SaasPro -- Curitiba, Brasil
            \switchcolumn
            \raggedleft 2022-2023
        \end{paracol}
    \end{onecolentry}
    \vspace{0.10cm}
    \begin{onecolentry}
        \begin{highlights}
                    \item Desenvolveu integrações backend utilizando Flask, FastAPI e Django.
                \item Implementou soluções com IA para engajamento e atendimento ao cliente.
        \end{highlights}
    \end{onecolentry}
    
\vspace{0.2cm}

    \begin{onecolentry}
        \setcolumnwidth{\fill, 4.5cm}
        \begin{paracol}{2}
            \textbf{Technical Support Engineer}, SaasPro -- Curitiba, Brasil
            \switchcolumn
            \raggedleft 2019-2023
        \end{paracol}
    \end{onecolentry}
    \vspace{0.10cm}
    \begin{onecolentry}
        \begin{highlights}
                    \item Atendeu clientes via múltiplos canais e integrou ferramentas via API com Python e Zapier.
                \item Aconselhou clientes sobre conformidade de e-mail e melhores práticas técnicas.
        \end{highlights}
    \end{onecolentry}
    
\section{Projetos}

    \begin{onecolentry}
        \textbf{Infraestrutura CI/CD Automatizada}\href{}{\, \faGithub}
    \end{onecolentry}
    \vspace{0.10cm}
    \begin{onecolentry}
        \begin{highlights}
                    \item Implementação de pipelines CI/CD com GitHub Actions, ambientes segregados e rollback seguro.
                \item Docker, GitHub Actions, Linux, Monitoramento.
        \end{highlights}
    \end{onecolentry}
    
\vspace{0.2cm}

    \begin{onecolentry}
        \textbf{Automação com IA para Atendimento ao Cliente}\href{}{\, \faGithub}
    \end{onecolentry}
    \vspace{0.10cm}
    \begin{onecolentry}
        \begin{highlights}
                    \item Criação de chatbots e fluxos automatizados usando IA para melhorar o engajamento.
                \item Python, Flask, FastAPI, Integrações com APIs, NLP.
        \end{highlights}
    \end{onecolentry}
    
\section{Competências Técnicas}
\begin{onecolentry}Java\end{onecolentry}
\vspace{0.2cm}
\begin{onecolentry}Python\end{onecolentry}
\vspace{0.2cm}
\begin{onecolentry}Flask\end{onecolentry}
\vspace{0.2cm}
\begin{onecolentry}FastAPI\end{onecolentry}
\vspace{0.2cm}
\begin{onecolentry}Django\end{onecolentry}
\vspace{0.2cm}
\begin{onecolentry}Docker\end{onecolentry}
\vspace{0.2cm}
\begin{onecolentry}Kubernetes\end{onecolentry}
\vspace{0.2cm}
\begin{onecolentry}Git\end{onecolentry}
\vspace{0.2cm}
\begin{onecolentry}GitHub Actions\end{onecolentry}
\vspace{0.2cm}
\begin{onecolentry}Microservices\end{onecolentry}
\vspace{0.2cm}
\begin{onecolentry}MySQL\end{onecolentry}
\vspace{0.2cm}
\begin{onecolentry}APIs REST\end{onecolentry}
\vspace{0.2cm}
\begin{onecolentry}Bash\end{onecolentry}
\vspace{0.2cm}
\begin{onecolentry}Linux\end{onecolentry}
\vspace{0.2cm}
\begin{onecolentry}Selenium\end{onecolentry}
\vspace{0.2cm}
\begin{onecolentry}Playwright\end{onecolentry}
\vspace{0.2cm}
\begin{onecolentry}Puppeteer\end{onecolentry}
\vspace{0.2cm}
\begin{onecolentry}CI/CD\end{onecolentry}
\vspace{0.2cm}
\begin{onecolentry}OpenShift\end{onecolentry}
\vspace{0.2cm}
\begin{onecolentry}GCP (conhecimento)\end{onecolentry}
\vspace{0.2cm}
\begin{onecolentry}Spring Boot (exposição)\end{onecolentry}
\section{Formação Acadêmica}

    \begin{onecolentry}
        \setcolumnwidth{\fill, 4.5cm}
        \begin{paracol}{2}
            \textbf{Universidade Federal de Santa Catarina (UFSC)}, Bacharelado em Tecnologias da Informação e Comunicação
            \switchcolumn
            \raggedleft Em andamento
        \end{paracol}
    \end{onecolentry}
    \vspace{0.10cm}
    \begin{onecolentry}
        \begin{highlights}
                    \item Monitor em Algoritmos e Programação com foco em Python; membro da Liga de IA e grupo de OffSec.
        \end{highlights}
    \end{onecolentry}
    
\vspace{0.2cm}

    \begin{onecolentry}
        \setcolumnwidth{\fill, 4.5cm}
        \begin{paracol}{2}
            \textbf{Universidade Federal do Paraná (UFPR)}, Bacharelado em Psicologia
            \switchcolumn
            \raggedleft Concluído em 2023
        \end{paracol}
    \end{onecolentry}
    \vspace{0.10cm}
    \begin{onecolentry}
        \begin{highlights}
                    \item Fundador da Liga Acadêmica de Neuropsicologia e participação em reabilitação cognitiva comunitária.
        \end{highlights}
    \end{onecolentry}
    
\section{Idiomas}
\begin{onecolentry}Português: Nativo\end{onecolentry}
\vspace{0.2cm}
\begin{onecolentry}Inglês: Fluente\end{onecolentry}
\end{document}