\documentclass[10pt, letterpaper]{article}

% Packages:
\usepackage[
    ignoreheadfoot,
    top=2 cm,
    bottom=2 cm,
    left=2 cm,
    right=2 cm,
    footskip=1.0 cm
]{geometry}
\usepackage{titlesec}
\usepackage{tabularx}
\usepackage{array}
\usepackage[dvipsnames]{xcolor}
\definecolor{primaryColor}{RGB}{0, 0, 0}
\usepackage{enumitem}
\usepackage{fontawesome5}
\usepackage{amsmath}
\usepackage[
    pdftitle={CV de Matheus Bruno De Moraes},
    pdfauthor={Matheus Bruno De Moraes},
    colorlinks=true,
    urlcolor=primaryColor
]{hyperref}
\usepackage[pscoord]{eso-pic}
\usepackage{calc}
\usepackage{bookmark}
\usepackage{lastpage}
\usepackage{changepage}
\usepackage{paracol}
\usepackage{ifthen}
\usepackage{needspace}
\usepackage{iftex}

% Ensure that generate pdf is machine readable/ATS parsable:
\ifPDFTeX
    \input{glyphtounicode}
    \pdfgentounicode=1
    \usepackage[T1]{fontenc}
    \usepackage[utf8]{inputenc}
    \usepackage{lmodern}
\fi

\usepackage{charter}

% Some settings:
\raggedright
\AtBeginEnvironment{adjustwidth}{\partopsep0pt}
\pagestyle{empty}
\setcounter{secnumdepth}{0}
\setlength{\parindent}{0pt}
\setlength{\topskip}{0pt}
\setlength{\columnsep}{0.15cm}
\pagenumbering{gobble}

\titleformat{\section}{\needspace{4\baselineskip}\bfseries\large}{}{0pt}{}[\vspace{1pt}\titlerule]

\titlespacing{\section}{
    -1pt
}{
    0.3 cm
}{
    0.2 cm
}

\renewcommand\labelitemi{$\vcenter{\hbox{\small$\bullet$}}$}
\newenvironment{highlights}{
    \begin{itemize}[
        topsep=0.10 cm,
        parsep=0.10 cm,
        partopsep=0pt,
        itemsep=0pt,
        leftmargin=0 cm + 10pt
    ]
}{
    \end{itemize}
}

\newenvironment{onecolentry}{
    \begin{adjustwidth}{
        0 cm + 0.00001 cm
    }{
        0 cm + 0.00001 cm
    }
}{
    \end{adjustwidth}
}

\newenvironment{header}{
    \setlength{\topsep}{0pt}\par\kern\topsep\centering\linespread{1.5}
}{
    \par\kern\topsep
}

\let\hrefWithoutArrow\href
\begin{document}

\begin{header}
    {\fontsize{25pt}{25pt}\selectfont Matheus Bruno De Moraes}

    \vspace{0pt}

    \normalsize
    \mbox{\href{tel:+55(41)995757856}{+55 (41) 9 9575-7856}} \enskip|\enskip \mbox{\href{mailto:mbrunomoraes97@gmail.com}{mbrunomoraes97@gmail.com}} \enskip|\enskip \mbox{Brazil} \\
    \mbox{\href{https://www.linkedin.com/in/brunomoraes97/}{LinkedIn}} \enskip|\enskip \mbox{\href{https://github.com/brunomoraes97}{Github}}
\end{header}

\vspace{5pt - 0.1cm}

\section{Resumo}
\begin{onecolentry}{Engenheiro de Suporte e DevOps experiente, com paixão por gerenciar todo o ciclo de vida do caso do cliente, desde o triage e reprodução até a resolução de problemas complexos. Proficiente em administração de sistemas Linux, scripting (Bash, Python), Git e CI/CD, com abordagem prática para resolver desafios técnicos e otimizar fluxos de trabalho. Capacidade comprovada de atuar como um elo fundamental entre as equipes de Produto e Engenharia, garantindo implementações suaves, comunicação rápida e alta satisfação do cliente. Comprometimento com a criação e melhoria de conteúdo de suporte, com ampla experiência em documentação técnica e tutoriais.}\end{onecolentry}
\section{Habilidades}
\begin{onecolentry}{\textbf{Administração de Sistemas Linux:} Ampla experiência com servidores Linux, Solução de problemas de desempenho, Ferramentas de diagnóstico, Contêineres Docker}\end{onecolentry}
\vspace{0.1cm}
\begin{onecolentry}{\textbf{DevOps e CI/CD:} Implementação de pipelines CI/CD com GitHub Actions, Gerenciamento de ambientes de staging e produção, Git e GitHub, Conhecimento de Openshift e Kubernetes}\end{onecolentry}
\vspace{0.1cm}
\begin{onecolentry}{\textbf{Linguagens de Programação:} Bash, Python, Java, C, Javascript, Ruby}\end{onecolentry}
\vspace{0.1cm}
\begin{onecolentry}{\textbf{Automação e Scripting:} Desenvolvimento de scripts de automação (Selenium, Playwright, Puppeteer), Otimização de processos, Integrações de API}\end{onecolentry}
\vspace{0.1cm}
\begin{onecolentry}{\textbf{Suporte Técnico e Experiência do Cliente:} Gerenciamento do ciclo de vida do caso do cliente, Comunicação técnica complexa, Solução de problemas (API, S2S, plataforma), Resolução de problemas complexos, Localização}\end{onecolentry}
\vspace{0.1cm}
\begin{onecolentry}{\textbf{Criação de Conteúdo:} Ampla experiência em escrita de conteúdo de suporte, Documentação técnica, Artigos da Base de Conhecimento, Tutoriais em vídeo, Guías de solução de problemas}\end{onecolentry}
\vspace{0.1cm}
\begin{onecolentry}{\textbf{Bancos de Dados:} MySQL (consultas, processamento de dados)}\end{onecolentry}
\vspace{0.1cm}
\begin{onecolentry}{\textbf{Segurança e Conformidade:} Compreensão de conceitos comuns de segurança da informação e conformidade (OffSec)}\end{onecolentry}
\vspace{0.1cm}
\begin{onecolentry}{\textbf{Ferramentas:} Git, GitHub, Postman, Zendesk, Jira, Zapier, N8N, CRMs semelhantes ao HubSpot}\end{onecolentry}
\section{Experiência Profissional}

\begin{onecolentry}
    \setcolumnwidth{\fill, 4.5cm}
    \begin{paracol}{2}
        \textbf{Gerente de Integração Técnica} \\ IREV, Limassol, Chipre (Remoto, tempo integral)
        \switchcolumn
        \raggedleft 2024-Atual
    \end{paracol}
\end{onecolentry}
\vspace{0.10cm}
\begin{onecolentry}
    \begin{highlights}
                \item Coordenador principal para integração de novos clientes
                \item Gerenciamento de configuração técnica, prazos e alinhamento de partes interessadas
                \item Participação em chamadas de escopo para avaliar a infraestrutura do cliente e garantir planejamento preciso de implementação
                \item Solução de problemas de integração de API, S2S e plataforma com equipes de produto e engenharia
                \item Padronização de documentação e aprimoramento de processos internos para maior eficiência de integração, contribuindo para o conteúdo de suporte
    \end{highlights}
\end{onecolentry}

\vspace{0.2cm}

\begin{onecolentry}
    \setcolumnwidth{\fill, 4.5cm}
    \begin{paracol}{2}
        \textbf{Engenheiro de Suporte Técnico L2 e Especialista em Automação} \\ Multilogin, Tallinn, Estônia (Remoto)
        \switchcolumn
        \raggedleft 2023-2024
    \end{paracol}
\end{onecolentry}
\vspace{0.10cm}
\begin{onecolentry}
    \begin{highlights}
                \item Investigação e resolução de problemas técnicos complexos, combinando habilidades analíticas e técnicas de solução de problemas criativas
                \item Desenvolvimento e implementação de scripts de automação usando Selenium, Playwright e Puppeteer para atender às necessidades dos clientes
                \item Trabalho com bancos de dados (MySQL), conduzindo consultas necessárias e realizando tarefas de processamento de dados conforme necessário
                \item Criação de documentação técnica e vídeos de demonstração para capacitar usuários e facilitar o autoatendimento, demonstrando experiência em escrita de conteúdo de suporte
                \item Experiência comprovada com cobertura de plantão, cobertura de fim de semana e turnos noturnos
    \end{highlights}
\end{onecolentry}

\vspace{0.2cm}

\begin{onecolentry}
    \setcolumnwidth{\fill, 4.5cm}
    \begin{paracol}{2}
        \textbf{Engenheiro de Software} \\ SaasPro, Curitiba, Brasil (Presencial)
        \switchcolumn
        \raggedleft 2022-2023
    \end{paracol}
\end{onecolentry}
\vspace{0.10cm}
\begin{onecolentry}
    \begin{highlights}
                \item Responsável por serviços profissionais fora do escopo do produto
                \item Desenvolvimento e manutenção de integrações entre diversas plataformas para otimização de processos de negócios
                \item Implementação de soluções de chatbot e IA para engajamento com o cliente
                \item Trabalho com frameworks web como Flask, FastAPI e Django
                \item Consultor técnico em chamadas de vendas e na estratégia de desenvolvimento
    \end{highlights}
\end{onecolentry}

\vspace{0.2cm}

\begin{onecolentry}
    \setcolumnwidth{\fill, 4.5cm}
    \begin{paracol}{2}
        \textbf{Engenheiro de Suporte Técnico} \\ SaasPro, Curitiba, Brasil (Presencial)
        \switchcolumn
        \raggedleft 2019-2023
    \end{paracol}
\end{onecolentry}
\vspace{0.10cm}
\begin{onecolentry}
    \begin{highlights}
                \item Fornecimento de suporte técnico excepcional aos clientes por meio de e-mail, bate-papo e chamadas, resolvendo problemas com eficiência e garantindo a satisfação do cliente
                \item Orientação de clientes sobre as melhores práticas e regulamentos de conformidade de e-mail
                \item Integração de diferentes ferramentas via API, utilizando código (Python) e ferramentas como Zapier e Make
                \item Colaboração com diversas equipes para atender às necessidades e feedbacks dos clientes, promovendo uma cultura centrada no cliente
    \end{highlights}
\end{onecolentry}

\section{Educação}

\begin{onecolentry}
    \setcolumnwidth{\fill, 4.5cm}
    \begin{paracol}{2}
        \textbf{Bacharelado em Tecnologias da Informação e Comunicação (TIC)} \\ Universidade Federal de Santa Catarina (UFSC), Brasil, Brasil
        \switchcolumn
        \raggedleft Atualmente Matriculado
    \end{paracol}
\end{onecolentry}
\vspace{0.10cm}
\begin{onecolentry}
    \begin{highlights}
                \item Assistente de ensino para a disciplina de Algoritmos e Programação (ênfase em Python), auxiliando os estudantes com lógica de programação
                \item Membro da Liga de Inteligência Artificial (LIA)
                \item Foco em POO, Java e Python
                \item Participação no grupo de extensão OffSec, aprofundando conhecimento em segurança da informação e conformidade
    \end{highlights}
\end{onecolentry}

\vspace{0.2cm}

\begin{onecolentry}
    \setcolumnwidth{\fill, 4.5cm}
    \begin{paracol}{2}
        \textbf{Bacharelado em Psicologia} \\ Universidade Federal do Paraná (UFPR), Brasil
        \switchcolumn
        \raggedleft Concluído em 2023
    \end{paracol}
\end{onecolentry}
\vspace{0.10cm}
\begin{onecolentry}
    \begin{highlights}
                \item Fundador e líder da Liga Acadêmica de Neuropsicologia, desenvolvendo habilidades de pesquisa e resolução de problemas
                \item Participação em projetos de extensão comunitária em reabilitação cognitiva
    \end{highlights}
\end{onecolentry}

\section{Idiomas}
\begin{onecolentry}{\textbf{Inglês:} Fluente}\end{onecolentry}
\vspace{0.1cm}
\begin{onecolentry}{\textbf{Português:} Nativo}\end{onecolentry}
\vspace{0.1cm}
\begin{onecolentry}{\textbf{Espanhol:} Intermediário}\end{onecolentry}
\end{document}