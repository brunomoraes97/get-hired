\documentclass[10pt, letterpaper]{article}

% Packages:
\usepackage[
    ignoreheadfoot,
    top=2 cm,
    bottom=2 cm,
    left=2 cm,
    right=2 cm,
    footskip=1.0 cm
]{geometry}
\usepackage{titlesec}
\usepackage{tabularx}
\usepackage{array}
\usepackage[dvipsnames]{xcolor}
\definecolor{primaryColor}{RGB}{0, 0, 0}
\usepackage{enumitem}
\usepackage{fontawesome5}
\usepackage{amsmath}
\usepackage[
    pdftitle={CV de Leonardo Schmidt},
    pdfauthor={Leonardo Schmidt},
    colorlinks=true,
    urlcolor=primaryColor
]{hyperref}
\usepackage[pscoord]{eso-pic}
\usepackage{calc}
\usepackage{bookmark}
\usepackage{lastpage}
\usepackage{changepage}
\usepackage{paracol}
\usepackage{ifthen}
\usepackage{needspace}
\usepackage{iftex}

% Ensure that generate pdf is machine readable/ATS parsable:
\ifPDFTeX
    \input{glyphtounicode}
    \pdfgentounicode=1
    \usepackage[T1]{fontenc}
    \usepackage[utf8]{inputenc}
    \usepackage{lmodern}
\fi

\usepackage{charter}

% Some settings:
\raggedright
\AtBeginEnvironment{adjustwidth}{\partopsep0pt}
\pagestyle{empty}
\setcounter{secnumdepth}{0}
\setlength{\parindent}{0pt}
\setlength{\topskip}{0pt}
\setlength{\columnsep}{0.15cm}
\pagenumbering{gobble}

\titleformat{\section}{\needspace{4\baselineskip}\bfseries\large}{}{0pt}{}[\vspace{1pt}\titlerule]

\titlespacing{\section}{
    -1pt
}{
    0.3 cm
}{
    0.2 cm
}

\renewcommand\labelitemi{$\vcenter{\hbox{\small$\bullet$}}$}
\newenvironment{highlights}{
    \begin{itemize}[
        topsep=0.10 cm,
        parsep=0.10 cm,
        partopsep=0pt,
        itemsep=0pt,
        leftmargin=0 cm + 10pt
    ]
}{
    \end{itemize}
}

\newenvironment{onecolentry}{
    \begin{adjustwidth}{
        0 cm + 0.00001 cm
    }{
        0 cm + 0.00001 cm
    }
}{
    \end{adjustwidth}
}

\newenvironment{header}{
    \setlength{\topsep}{0pt}\par\kern\topsep\centering\linespread{1.5}
}{
    \par\kern\topsep
}

\let\hrefWithoutArrow\href
\begin{document}

    \begin{header}
        {\fontsize{25pt}{25pt}\selectfont Leonardo Schmidt}

        \vspace{5pt}

        \normalsize
        \mbox{Brasil / SC} \textbar{} \mbox{\href{mailto:leo.schmidt@email.com}{leo.schmidt@email.com}} \textbar{} \mbox{\href{tel:(48)998765432}{(48) 99876-5432}} \textbar{} \mbox{\href{https://leoschmidt.ai}{leoschmidt.ai}} \textbar{} \mbox{\href{https://linkedin.com/in/leonardo-schmidt-ds}{linkedin.com/in/leonardo-schmidt-ds}} \textbar{} \mbox{\href{https://github.com/leoschmidt}{github.com/leoschmidt}}
    \end{header}

    \vspace{5pt - 0.3cm}
    
\section{Experiência Profissional}

    \begin{onecolentry}
        \setcolumnwidth{\fill, 4.5cm}
        \begin{paracol}{2}
            \textbf{Cientista de Dados Sênior}, Nexus Data Analytics -- Florianópolis, SC
            \switchcolumn
            \raggedleft Fev 2022 – Presente
        \end{paracol}
    \end{onecolentry}
    \vspace{0.10cm}
    \begin{onecolentry}
        \begin{highlights}
                    \item Desenvolvi e implementei modelos de previsão de demanda que reduziram os custos de estoque em 18%.
                \item Liderei a arquitetura de um pipeline de MLOps na AWS (S3, SageMaker, Lambda), automatizando o re-treinamento e deploy de modelos.
                \item Criei dashboards interativos em Power BI para apresentar resultados e KPIs dos modelos para stakeholders, melhorando a tomada de decisão.
        \end{highlights}
    \end{onecolentry}
    
\vspace{0.2cm}

    \begin{onecolentry}
        \setcolumnwidth{\fill, 4.5cm}
        \begin{paracol}{2}
            \textbf{Cientista de Dados}, VarejoConect -- Remoto
            \switchcolumn
            \raggedleft Jul 2019 – Jan 2022
        \end{paracol}
    \end{onecolentry}
    \vspace{0.10cm}
    \begin{onecolentry}
        \begin{highlights}
                    \item Construí um sistema de recomendação de produtos (filtragem colaborativa) que aumentou o cross-selling em 12%.
                \item Realizei análises de clusterização (K-Means) para segmentação de clientes, direcionando campanhas de marketing personalizadas.
                \item Automatizei a coleta e limpeza de dados de múltiplas fontes usando Python (Pandas) e SQL, reduzindo o tempo de preparação de dados em 75%.
        \end{highlights}
    \end{onecolentry}
    
\section{Formação Acadêmica}

    \begin{onecolentry}
        \setcolumnwidth{\fill, 4.5cm}
        \begin{paracol}{2}
            \textbf{Universidade Federal de Santa Catarina (UFSC)}, Mestrado em Ciência da Computação
            \switchcolumn
            \raggedleft 2017 – 2019
        \end{paracol}
    \end{onecolentry}
    \vspace{0.10cm}
    \begin{onecolentry}
        \begin{highlights}
                    \item Foco em Inteligência Artificial e Aprendizado de Máquina.
                \item Dissertação: "Detecção de Anomalias em Séries Temporais Financeiras com Redes Neurais Recorrentes (LSTM)".
        \end{highlights}
    \end{onecolentry}
    
\vspace{0.2cm}

    \begin{onecolentry}
        \setcolumnwidth{\fill, 4.5cm}
        \begin{paracol}{2}
            \textbf{Universidade do Estado de Santa Catarina (UDESC)}, Bacharelado em Sistemas de Informação
            \switchcolumn
            \raggedleft 2013 – 2016
        \end{paracol}
    \end{onecolentry}
    \vspace{0.10cm}
    \begin{onecolentry}
        \begin{highlights}
                    \item Iniciação Científica na área de otimização de consultas em bancos de dados.
        \end{highlights}
    \end{onecolentry}
    
\section{Projetos de Destaque}

    \begin{onecolentry}
        \textbf{API de Classificação de Sentimentos}\href{https://github.com/leoschmidt/sentiment-api}{\, \faGithub}
    \end{onecolentry}
    \vspace{0.10cm}
    \begin{onecolentry}
        \begin{highlights}
                    \item Criei uma API RESTful com FastAPI para classificar textos em tempo real (positivo, negativo, neutro) usando um modelo treinado com Scikit-learn.
                \item Empacotei a aplicação com Docker para facilitar o deploy em qualquer ambiente de nuvem.
                \item Ferramentas: FastAPI, Docker, Scikit-learn, NLTK
        \end{highlights}
    \end{onecolentry}
    
\vspace{0.2cm}

    \begin{onecolentry}
        \textbf{Análise de Corrupção no Brasil (Dados Abertos)}\href{https://github.com/leoschmidt/analise-corrupcao-br}{\, \faGithub}
    \end{onecolentry}
    \vspace{0.10cm}
    \begin{onecolentry}
        \begin{highlights}
                    \item Notebook de análise exploratória de dados públicos sobre gastos governamentais, utilizando Pandas, Matplotlib e Seaborn para gerar visualizações.
                \item Identifiquei padrões e outliers em licitações e contratos públicos.
        \end{highlights}
    \end{onecolentry}
    
\section{Publicações}

    \begin{onecolentry}
        \setcolumnwidth{\fill, 4.5cm}
        \begin{paracol}{2}
            \textbf{A Robust Framework for Anomaly Detection in Financial Time Series}
            \switchcolumn
            \raggedleft Maio 2019
        \end{paracol}
    \end{onecolentry}
    \vspace{0.10cm}
    \begin{onecolentry}
        \mbox{Dr. Helena Borges}, \mbox{\textbf{\textit{Leonardo Schmidt}}}
        \vspace{0.10cm}
        \href{https://doi.org/10.1109/ICDM.2019.00123}{10.1109/ICDM.2019.00123}
    \end{onecolentry}
    
\end{document}